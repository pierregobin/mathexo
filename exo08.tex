\documentclass{article}
\usepackage{ae,lmodern}
\usepackage[francais]{babel}
\usepackage[utf8]{inputenc}
\usepackage[T1]{fontenc}
\usepackage{tikz,tkz-tab}
\usepackage{multicol}
\usepackage{xcolor}

\usepackage[a4paper,top=1cm,bottom=1cm,left=3cm,right=3cm]{geometry}
\usepackage{amsmath}
\usepackage{amsfonts}
\usepackage{amssymb}

\newcounter{numexos}
\setcounter{numexos}{0}
\newcommand{\exercice}[1]{
\addtocounter{numexos}{1}
\textcolor{red}{Exercice\,\thenumexos\,:}\,#1
\\
}

\newcounter{solexos}
\setcounter{solexos}{0}
\newcommand{\solution}[1]{
\addtocounter{solexos}{1}
\textcolor{red}{Correction\,\thesolexos\,:}\,#1
}


\newcommand{\variation}[1]{
\exercice{calculer la dérivée de 
$f$ , déterminer les valeurs qui annulent $f'$ , en déduire les extremums et les variations de $f$.\,#1
}
}

\begin{document}
\title{variations}
\section{Enoncés des exercices}

\variation{$f(t) = -\dfrac{t}{9} + 4 + \ln(t)$}
\variation{$f(t) = (8t+64)*exp(-t/8)$}
\variation{$f(t) = -t + 1 + ln(-2t + 8)$}
\variation{$f(t) = t^2+14t+5$}
\variation{$f(t) = (6t+48)exp(\dfrac{t}{6})$}
\variation{$f(t)=\dfrac{t}{3} + 7 + \ln(t)$}

\section{corrigés}
%  \begin{multicols}{2}
  \solution{On détermine le domaine de définition de $f$. $f(t) = -\dfrac{t}{9} + 4 + \ln(t)$. $f$ est définie pour $t > 0$ à cause de $\ln$. Donc $D_f = \mathbb{R}^{+*}$.  $f$ est dérivable sur $D_f$ et sa dérivée $f'(t) = -\dfrac{1}{9} + \dfrac{1}{t}$. La dérivée s'annule en $t=9$.\\
  \begin{tikzpicture}
     \tkzTabInit{$t$ / 1, $f'(t)$ / 1 , $f$ / 1}{$0$, $9$, $+\infty$}
     \tkzTabLine{d,+,z,-,}
     \tkzTabVar{-/ $-\infty$, +C/ $3+\ln(9)$, -/ $-\infty$}
\end{tikzpicture}
  }
  
  \solution{Le domaine de définition de $f$ est $\mathbb{R}$.
  Sa dérivée : $f'(t) = 8.exp(-\dfrac{t}{8}) + (8t+64).(-1/8).exp(-\dfrac{t}{8})$ et après simplification : $f'(t) =-t exp(-\dfrac{t}{8})$. $f'$ est s'annule en $0$, est positive sur
$\mathbb{R}^{*-}$ et négative sur $\mathbb{R}^{*-}$.
\\
  \begin{tikzpicture}
     \tkzTabInit{$t$ / 1, $f'(t)$ / 1 , $f$ / 1}{$-\infty$, $0$, $+\infty$}
     \tkzTabLine{,+,z,-,}
     \tkzTabVar{-/ $-\infty$, +C/ $64$, -/ $-\infty$}
\end{tikzpicture}


\solution{La fonction $f(t) = -t + 1 + ln(-2t + 8)$ est définie lorsque $-2t+8 > 0$, soit $t < -4$. $D_f =]-\infty,-4[$. Elle est dérivable sur $D_f$. Sa dérivée vaut : $f'(t) = -1 + -\dfrac{2}{-2t+8}$ après simplification, $f'(t) = \dfrac{t-5}{-t+4}$. $f'$ est négative sur $D_f$, $f$ est donc décroissante sur $D_f$.}
  }

\solution{La fonction $f(t) = t^2+14t+5$ est définie sur $\mathbb{R}$ tout entier. Sa dérivée est $f'(t) = 2t+14$, elle s'annule en $-7$.\\
  \begin{tikzpicture}
     \tkzTabInit{$t$ / 1, $f'(t)$ / 1 , $f$ / 1}{$-\infty$, $-7$, $+\infty$}
     \tkzTabLine{,-,z,+,}
     \tkzTabVar{
     +/ $+\infty$, -C/ $-44$, +/ $+\infty$}
\end{tikzpicture}

}

\solution{La fonction $f(t) = (6t+48)\exp(\dfrac{t}{6})$ est définie sur $\mathbb{R}$. Elle est dérivable, sa dérivée est : $f'(t) = 6 \exp(\dfrac{t}{6}) + (6t+48)\dfrac{1}{6}\exp(\dfrac{t}{6})$, soit $f'(t) = (t+14)\exp(\dfrac{t}{6})$. $f'$ s'annule en $-14$, elle est négative pour $t < -14$ et positive pour $t > -14$.\\
  \begin{tikzpicture}
     \tkzTabInit{$t$ / 1, $f'(t)$ / 1 , $f$ / 1}{$-\infty$, $-14$, $+\infty$}
     \tkzTabLine{,-,z,+,}
     \tkzTabVar{
     +/ , -C/ $-36\exp(\dfrac{-7}{3})$, +/}
\end{tikzpicture}


}

\solution{
La fonction $f(t)=\dfrac{t}{3} + 7 + \ln(t)$. Son domaine de définition est $D_f = \mathbb{R}^{*+}$. Sa dérivé est $f'(t)=\dfrac{1}{3}+\dfrac{1}{t}$. Sur $D_f$, $f'$ est strictement positive, donc $f$ est croissante sur $\mathbb{R}^{*+}$. }


%  \end{multicols}
\end{document}

