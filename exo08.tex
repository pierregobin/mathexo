\documentclass{article}
\usepackage{ae,lmodern}
\usepackage[francais]{babel}
\usepackage[utf8]{inputenc}
\usepackage[T1]{fontenc}
\usepackage{tikz,tkz-tab}
\usepackage{multicol}
\usepackage{xcolor}

\usepackage[a4paper,top=1cm,bottom=1cm,left=3cm,right=3cm]{geometry}
\usepackage{amsmath}
\usepackage{amsfonts}
\usepackage{amssymb}

\newcounter{numexos}
\setcounter{numexos}{0}
\newcommand{\exercice}[1]{
\addtocounter{numexos}{1}
\textcolor{red}{Exercice\,\thenumexos\,:}\,#1
\\
}

\newcounter{solexos}
\setcounter{solexos}{0}
\newcommand{\solution}[1]{
\addtocounter{solexos}{1}
\textcolor{red}{Correction\,\thesolexos\,:}\,#1
}


\newcommand{\variation}[1]{
\exercice{calculer la dérivée de 
$f$ , déterminer les valeurs qui annulent $f'$ , en déduire les extremums et les variations de $f$.\,#1
}
}

\begin{document}
\title{variations}
\section{Enoncés des exercices}

\variation{$f(t) = -\dfrac{t}{9} + 4 + \ln(t)$}
\variation{$f(t) = (8t+64)*exp(-t/8)$}
\variation{$f(t) = -t + 1 + ln(-2t + 8)$}
\variation{$f(t) = t^2+14t+5$}
\variation{$f(t) = (6t+48)exp(\dfrac{t}{6})$}
\variation{$f(t)=\dfrac{t}{3} + 7 + \ln(7)$}

\section{corrigés}
%  \begin{multicols}{2}
  \solution{On détermine le domaine de définition de $f$. $f(t) = -\dfrac{t}{9} + 4 + \ln(t)$. $f$ est définie pour $t > 0$ à cause de $\ln$. Donc $D_f = \mathbb{R}^{+*}$.  $f$ est dérivable sur $D_f$ et sa dérivée $f'(t) = -\dfrac{1}{9} + \dfrac{1}{t}$. La dérivée s'annule en $t=9$.\\
  \begin{tikzpicture}
     \tkzTabInit{$t$ / 1, $f'(t)$ / 1 , $f$ / 1}{$0$, $9$, $+\infty$}
     \tkzTabLine{d,+,z,-,}
     \tkzTabVar{-/ $-\infty$, +C/ $3+\ln(9)$, -/ $-\infty$}
\end{tikzpicture}
  }
  
  \solution{}
%  \end{multicols}
\end{document}

